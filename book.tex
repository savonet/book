\documentclass{book}
\usepackage[a5paper]{geometry}
% \usepackage{avant}

\renewcommand{\familydefault}{\sfdefault}
\renewcommand{\labelitemi}{--}

\title{The Liquidsoap book}
\author{David Baelde \and Romain Beauxis \and Samuel Mimram}

\begin{document}
\maketitle

\tableofcontents

\chapter{Introduction}
web radio, real life examples, programming languages, how can they make it
better than a configuration file

\chapter{Installation}
How to install the program?

\chapter{Setting up a simple radio station}

\chapter{The language}
Basics of functional programming for liquidsoap
Syntax of liquidsoap language

\section{Sources}
\subsection{What is a faillible source?}

\section{Stings and logging}

\section{Functions}

\section{Interaction}
interactive floats

\section{The typing system}

\chapter{Full workflow of a radio station}
\section{Playlists}

inotify

\section{Interactive playlists}

\section{Handling tracks}
\begin{itemize}
\item crossfade
\item blank detection
\end{itemize}

\section{Signal processing}

\section{Input streams with harbor}

\chapter{Specific use cases}
\section{Using command-line arguments}
shebang, argv

\section{Dealing with HTTP requests}

\section{External scripting}

\section{Decoders}

\section{External decoders/encoders}

\section{Reading files}

\chapter{Plugins}
\section{GStreamer}

\chapter{Video}

\section{Frei0r}

\chapter{The Liquidsoap ecosystem}
\section{Flows}

\section{Aitime}

\section{Online services}

\chapter{Internals}
\section{The OCaml language}

\section{The source model}

\section{How to contribute}

\chapter{Conclusion}
\end{document}
