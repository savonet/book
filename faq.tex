\chapter{Frequently asked questions}
reprendre la \href{https://www.liquidsoap.info/doc-dev/faq.html}{FAQ DU SITE}

\section{Type errors}
\Liquidsoap might reject a script with a series of errors of the form \code{this
  value has type ... but it should be a subtype of ...}. Usually the last error
tells you what the problem is, but the previous errors might provide a better
information as to where the error comes from.

For example, the error might indicate that a value of type \code{int} has been
passed where a float was expected, in which case you should use a conversion, or
more likely change an integer value such as \code{13} into a float
\code{13.}.

A type error can also show that you're trying to use a source of a certain
content type (e.g., audio) in a place where another content type (e.g., pure
video) is required. In that case the last error in the list is not the most
useful one, but you will read something like this above:
\begin{verbatim}
At ...:
  this value has type
    source(audio=?A+1,video=0,midi=0)
    where ?A is a fixed arity type
  but it should be a subtype of
    source(audio=0,video=1,midi=0)
\end{verbatim}

Sometimes, the type error actually indicates a mistake in the order or labels of
arguments. For example, given \code{output.icecast(mount="foo.ogg",source)}
liquidsoap will complain that the second argument is a source
(\code{source(?A)}) but should be a format (\code{format(?A)}): indeed, the
first unlabelled argument is expected to be the encoding format, e.g.,
\code{\%vorbis}, and the source comes only second.

Finally, a type error can indicate that you have forgotten to pass a mandatory
parameter to some function. For example, on the code
\code{fallback([crossfade(x),...])}, liquidsoap will complain as follows:
\begin{verbatim}
At line ...:
  this value has type
    (?id:string, ~start_next:float, ~fade_in:float,
     ~fade_out:float)->source(audio=?A,video=?B,midi=0)
    where ?B, ?A is a fixed arity type
  but it should be a subtype of
    source(audio=?A,video=?B,midi=0)
    where ?B, ?A is a fixed arity type
\end{verbatim}
Indeed, \code{fallback} expects a source, but \code{crossfade(x)} is
still a function expecting the parameters \code{start\_next},
\code{fade\_in} and \code{fade\_out}.


\section{Unable to decode “file” as \code{\{audio=2;video=0;midi=0\}!}}

\section{That source is fallible!}

\section{Clock errors}

\section{We must catchup}

\section{Exceptions}